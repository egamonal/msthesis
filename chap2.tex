\chapter{Related Work}
This chapter is a short introduction to.. general concepts. doesnt make references to
robots, qt or anything. no ref to UX or UI design. my problem is an app that renders
apps, not the screens editor or a particular design.

\section{Reengineering}
TODO: two options here. either include the  quote by pressman or rethink it.
\begin{quote} 
Consider any technology product that has served you well. 
You use it regularly, but it's getting old. 
It breaks too often, takes longer to repair than you'd like, and no longer represents the newest technology.
What to do? If the product is hardware, you'll likely throw it away and buy a newer model.
But if it's custom-built software, that option may not be available. 
You'll need to rebuild it. 
You'll create a product with added functionality, better performance and reliability, and improved maintainability.
That's what we call reengineering. \cite{Pressman:2007}
\end{quote}

Unlike other products, software does not degrade with time due to external factors like the inclemencies of the weather, power outages or intense use.
Software packages are adapted, corrected, extended and improved constantly and that can lead to unstable applications or unexpected side effects, even if the original product had been designed with the best practices at that time.
Documentation gets outdated, after a certain amount of time the specification does not match the product any more and changes are harder to implement . % FIXME language
Using frameworks, libraries and third-party software is a common practice. Sometimes the vendor of the technology used in a project stops supporting it and the only solution is substituting the components that depend on it. 
The product becomes unmaintainable in many aspects and reenginering is required. 

Software reengineering comprises 6 activities: inventory analysis, document reestructure, reverse engineering, code restructuring, data restructuring and forward engineering.
Simple put, it is all about understanding the current logic of a program and modifying the three typical components: data, logic and view.
The project of this thesis is not modifying the current program but developing a new one with new technology, focusing on reverse engineering the original program. 

introduction

software reenginering:
main activities (software reengineering prcess model defines 6 activities, fig 30.2):
inventory analysis
document reestructure (option 1)
reverse engineering
code restructuring
data restructuring
forward engineering

reverse engineering:
keys: abstraction level, completeness, directionality:
abstraction: make it as high as possible (ideally, entity-relationship models)
completeness: how much detail prvided at an abstraction level?
(there are tools to analyse software but this thesis isn't using them)
directionality: one way. extract info from the source code, give it to an engineer, use it during any mmaintenance activty,. it can also be two-way (feed info to a reengineering tool that fixes the old program)

first activity of reverse engineering: understand processing. analyse code at varying levels of abstraction: system, program, component, pattern, statement. 
understand overall functionality before doing more steps (set a context, pay attention to interoperabilty issues). Block diagrams of the programs (show interaction). do some narrative for each component.
normally there is code for: prepare data, process data, prepare results for the view

next activity: understand data:
internal data structures. goal: identify classes. focus on teh definition of classes of objects. book is old-fashioned and assumes crappy old code.
database structure. bottomline: rethink the design.

next: reverse enginer UI

restructuring
(mention it. this is related to TDD)
code restructuring is not enough. it alleviates immediate problems. real benefit is achieved only when data and architecture are restrucured. spaguetti-bowl code to structured programming.
data restructuring: standarise, rationalise, make physical modifications.

forward engineering: change the code, redesign, or make big changes (completely redesign, recode, test..). preventive maintenance. cost of maintenance.


\section{Web Technologies}
% quotations: something about emacs life style, or berners lee
The \ac{WWW} plays a central role in many people's lives.
The first drafts date back to the 1980s, the first web browser prototype by Tim Berners-Lee was built in December 1990 and the Mosaic browser was completed in 1993.
After that, Netscape, Mozilla Firefox, Microsoft Internet Explorer, Google Chrome, Safari, Opera, Konqueror and many other browsers appeared, along with some nowadays popular websites (e.g. amazon.com 1995).

For many people the browser is the most used software in many devices. 
The web creates opportunities and makes documents accessible like tax records, maps, banking information, phone directories or product catalogues.
New business models, like new ways of promoting products, selling music or movies, are developed having the web as a core part.
Innovation goes beyond all these with experiments and products like FirefoxOS or ChromeOS to build an operating system designed to mainly let the user browse the web. 
Other companies, like Citrix or eyeOS, offer web applications that resemble an operating system %\cite{Gamonal:2011}.

%goals: explain evolution of web tech, state of the art, and why it is better than flash.

%* programs in browsers. isolate them? flash plugin. flash is dead (jobs or other
%articles) 
%* browsers: architecture (dom, js engine, html renderer...) from mosaic to
%%fx/ch/qtwebkit 
%** webkit and qtwebkit
%
%* client-side tech: JS, css, html/xml 
%** evaluation and comparison 
%** frameworks
%
%browsers, JS engines, chrome v8, fx gecko, QtBrowser
% functionaliyy (interoperability, security), reliability (maturity), usability: (learn-
%ability, attractiveness) Efficiency: (time behavior, resource utilization) Maintainabil-
%ity: (stability) Portability: (installability)

\subsection{From Plain Documents to Applications}
\paragraph{The Evolution of the Web}
"HyperText is a way to link and access information of various kinds as a web of nodes in which the user can browse at will. Potentially, HyperText provides a single user-interface to many large classes of stored information such as reports, notes, data-bases, computer documentation and on-line systems help" \cite{BernersLee:1990}

When the \ac{WWW} was started in 1990 it was intended to be a document viewing platform.
It has evolved in at least three phases \cite{Anttonen:2011} \cite{Taivalsaari:2008}:

In the early days, websites were basically a few files with plain text, forms and page-structured artifacts, possibly static images and hyper-links (anchors) to provide navigation. 

Later on, browsers added programming capabilities, plug-ins and allowed animated graphics. Web pages became interactive and client-server communication increased in complexity.
Adobe Flash (at that time Macromedia Flash), ShockWave, Java, QuickTime and a few others allowed the inclusion of multimedia contents or even programs (e.g. Java Applets).
Engineers could combine server-side and client-side technology to provide some more capabilities along with content.

More recently, despite the fact that the web browser was not designed for running applications, since 2008 users have headed this way. 
In this third phase, web pages are not just documents: they have complex user interaction capabilities, they do not require full page refresh and typical use cases have evolved. 
A big part of applications is executed client-side, taking advantage of the computing capabilities of modern computers and decreasing the load in servers. In some cases, the application also needs huge server-side resources and uses strategies like cloud computing (e.g. Amazon EC2) to scale up.
Today the web is not only about viewing documents but about world-wide sharing, collaboration and interaction, possibly in real time. 

Web browsers have become a widely-used platform for software applications. 
Video editors, spreadsheets, calendars or 3D games used to run exclusively on desktop computers. 
Today, they run in a variety of devices and some of them even do it in a browser. 
There are 3D engines ported to Web, real time collaboration, full HD video support without plug-ins and many more features built-in a system, the browser, that is not an ideal execution environment for desktop applications.

\paragraph{\acp{RIA}}
\acs{RIA} are neither web services or web pages. 
They are software systems based on technologies and standards of the \ac{W3C} that provide web specific resources such as content and services through a web browser \cite{Kappel:2006}.
Typically one designs them as single page websites.

% FIXME language
Web applications differ a lot between them. 
They can be document-centered, workflow-based, portal-oriented, collaborative, social, etc. 
They all have some common characteristics. Amongst others:
\begin{itemize}
    \item The product
    \begin{itemize}
        \item Content is the core.
        \item Hyper-text: non-linearity is a main distinction to traditional software systems. There are many ways of landing on a certain page. This can lead to disorientation or cognitive overload for users.
        \item Presentation aesthetics, usability and interaction are closer to a desktop application than to a web page
    \end{itemize}
    \item Use
    \begin{itemize}
        \item Globality: Spontaneity and multiculturality of users. Web applications are publicly available.
        \item Quality suffers from unknown network characteristics
        \item Multiplaform delivery involves having different devices, browsers and degrees of functionality and performance
        \item Intense network usage. Remote calls are to be minimised. The use of software patterns like remote facade, remote proxy, DTO or RPC is frequent
    \end{itemize}
    \item Evolution
    \begin{itemize}
         \item Continuous change
         \item Competitive pressure
         \item Fast pace development
    \end{itemize}   
\end{itemize}

The project of this thesis is, in a sense, both a \ac{RIA} and a \acp{RIA} generator. 
It is a \ac{RIA} because, among other reasons, it works in the browser with web technology and has an intense use of the network to fetch the model or other components to build the screens.
The rendered application works in a browser, has an intense use of the network as well and is content-centered.
Despite the fact that it works in the specific content of a robot, globality is still an issue. A typical scenario would be a fleet of robots in a conference: users from many different cultural backgrounds can use the application at any time.



%    * from old plain html to RIA 
%    * browse docs to single page apps

\subsection{Browsers and isolation of programs}    
Many of the websites in the aforementioned third phase contain substantial amounts of client-side code. At the end of the day, a \ac{RIA} is a distributed software. 
In the old days the client-side part was thin and simple, whereas today's application have complex logic delegated to client nodes with local storage, hardware-accelerated components, websockets and other advanced capabilites.

In spite of the fact that \acp{RIA} have grown in complexity and now demand more resources, browsers architectures still do not provide sufficient isolation between concurrently executing programs.
A similar problem occurred in early operating systems (e.g. MS-DOS), before processes appeared \cite{Reis:2009}. 

Browsers have typically the following components: TODO

Most of browsers have a monolithic architecture with poor isolation between web application instances. Chromium, however, implemented an architecture based on OS processes. 
Another way of isolating web applications is running them in a plug-in container. 

The project of this thesis uses Qt WebKit, the port of WebKit on top of Qt, to display the application in a Qt dialog and communicate with the hardware. 
Qt WebKit components comprise the WebCore and SquirrelFish Extreme, which compiles Javascript into native machine code. 
It is compatible with Adoble Flash Player but because it will not receive more updates in the future, it will not be used any more.
The Flash plugin is a black box isolated from other elements in the \ac{DOM}.
Isolation is not an issue in this project because there is only one application running at a time.

\section{Client-side Technology}


\subsection{Comparison of languages}
\subsection{Frameworks}

\section{Test-Driven Development}
The software written for this thesis is executed in a complex environment.
Testing it properly and ensuring the quality is crucial to avoid unexpected behaviour in one of the most visible parts of this system, the screen.
Crashes and \ac{UI} flaws can lead to triggering wrong external calls to other components of the Qt system (e.g. displaying a Qt dialog different than expected, not making the call, sending it with bad parameters...), blocking access to features or seriously affecting the \ac{UX}.

Reengineering this system has, at least, two sources of potential errors: 
undocumented features in the original application (unknown, documented wrongly, partially documented or not described at all) and integration of the new software in the current system.

% maybe put all this in a table?
\ac{TDD} is based on having numerous and very short iterations with these steps:
\begin{enumerate}
    \item Add an initial (failing) test
    \item Run all tests and see if the new one fails
    \item Write the minimum amount of code to pass the test
    \item Run the automated tests and see them succeed
    \item Refactor the code. Conform to standards and best practices
\end{enumerate}

As opposed to classic development methodologies (e.g. waterfall, where test are done in the last place), with \ac{TDD} tests go first. 
Specification and documentation of the system are not artifacts created before the implementation but the tests themselves.

Testing earlier and heavily has benefits over the classical approach:

\textbf{Efficiency} Defects (and their causes) are identified earlier.

\textbf{Robustness} The system is more reliable and stable. 
The \ac{QA} process is easier to maintain and much more strict.

\textbf{Maintainability} Small fixes can introduce new errors in the system. 
Testing continuously and producing the minimum amount of code to implement a feature reduces this.

\textbf{Extensibility} \ac{XP} embraces change. 
It is easier to adapt to changes in the requirements. 
This thesis is reengineering a project that is not frozen. 
During the development, the original software keeps adding features.

This methodology is specially useful in the implementation of this project for two reasons: 
the Google AngularJS framework is extremely friendly with unit testing and \ac{E2E} tests thanks to Karma Runner, and the company has the infrastructure to incorporate the project in a continuous integration system with Jenkins.

\section{XML as an intermediate representation}


