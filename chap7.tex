\chapter{Implementation}
\label{chap:implementation}
This chapter describes the environment to develop and run Flango \cm.
It illustrates the implementation of key features with code examples.

\section{Development Environment Set-Up}
Flango \cm belongs to a bigger project, Flango and the robot's system.
The development environment is adapted to this situation to ensure a correct integration.
Moreover, the project uses tools to run unit tests, generate code, etc.

\subsection*{Integration}
The files of Flango \cm live in the server as part of the \flangobe .
The backend is developed with Django and requires a set of Python libraries that might not be available in the system.
To overcome this situation, Flango is developed in a virtual environment and files are made available to clients with the built-in web server in Django\footnote{\url{http://www.djangoproject.com}}, although Basestation uses Apache2\footnote{\url{http://httpd.apache.org}}, and the robot Lighttpd\footnote{\url{http://www.lighttpd.net}}.
All the code is in a subversion\footnote{\url{http://subversion.tigris.org}} repository.

\begin{lstlisting}[caption=Creation of virtual environments, label=virtual-env]
$ mkvirtualenv local-flango
$ workon local-flango
$ pip install --upgrade -r requirements.txt
\end{lstlisting}

Virtual environments have pre and postactivate scripts to alter the default behaviour and can have versions of Python and other packages different than the system's.
Flango needs to modify environment variables have \texttt{local-flango} in the local copy of the repository, use a database, set aliases, etc.

After the virtual environment is ready, it is easier to switch development branches (e.g. working on two features at the same time, but using different versions or different databases to avoid conflicts and guarantee that they both have a safe starting point).
Django is a web framework for Python and has a built-in development web server that runs on port 8000.

\begin{lstlisting}[caption=Django web server, label=virtual-env-server]
~/contentMgmtBranches/trunk/backend$ django runserver
Validating models...

modeltranslation: Registered 0 models for translation () [pid: 5286].
0 errors found
Django version 1.4.3, using settings 'flango.settings'
Development server is running at http://127.0.0.1:8000/
Quit the server with CONTROL-C.

"GET /static/flangoh/flangoh/app/index.html HTTP/1.1" 200 2317
\end{lstlisting}

The file structure of the whole project is shown in listing \ref{impl-flango-files-layout}.
Folders \texttt{build}, \texttt{packaging} and \texttt{conf} contain scripts that are used when building Debian packages that will deploy the software on the robot and basestation.
\texttt{src} contains the \flangobe , which includes the \flangofe (and the \se).
The project of this thesis, Flango \cm is in the \texttt{static} folder (listing \ref{impl-flango-static-files-layout}), that contains files served without any server-side processing.

\begin{lstlisting}[columns=fixed,caption=Flango files layout, label=impl-flango-files-layout]
~/contentMgmtBranches/trunk/backend$ ls
build  conf  packaging  requirements.txt  scripts
src    static
\end{lstlisting}

The browser can open any file in the \texttt{static} folder.
Alternating between the new and the old version only requires to load \texttt{frontend/index.html} (\flash) or \\ \texttt{flangoh/flangoh/app/index.html} (Angular).

\begin{lstlisting}[columns=fixed,language=bash,caption=Flango static files layout, label=impl-flango-static-files-layout]
~/contentMgmtBranches/trunk/backend/static$ ls
admin       flangoh   imports   vpnUsersStatus.xml
css         fonts     js        vpnUsers.xml
exports     frontend  lib
flango-gui  img       media
\end{lstlisting}

The development computer also has a working copy of the code for the robot and Basestation in order to run simulations and to test the new \cm  without using robot time, a scarce resource.

There are 3 run levels (each with a matching executable file):
\begin{enumerate}
\item \textbf{htmlDialog}. A basic Qt dialog with a QtWebKit widget.
\item \textbf{guiServer}. A version with more features enabled, like the sound server, navigation, etc.
\item \textbf{stateMachine}. A full program with a testing state machine, Qt, sound server, etc.
\end{enumerate}

An \ac{XML} file (\texttt{robot/sources/etc/reemh3/robot.xml}) defines the \ac{URL} to load.
Listing \ref{impl-robot-xml} shows a part of the configuration file.
In this case, the robot (or the simulation) loads the old version, an \ac{HTML} file in \texttt{http://localhost/static/frontend/index.html} that only contains a full-screen \flash object, the old Flango \cm .
\begin{lstlisting}[language=xml,caption=Robot configuration file, label=impl-robot-xml]
<controller type="Gui" implementation="" debug="1">
    <guiController>
    <theme>barcelona</theme>
    <timeout>45</timeout>
    <fullHtml>1</fullHtml><!--  full screen  -->
    <views>
        ...
        <view
            task="welcome" 
            args="http://localhost/static/frontend/index.html" 
            title="" 
            runMeetPeople="1"  
            activateASR="root/pal/reem-functions" />
...    
\end{lstlisting}

\begin{lstlisting}[language=bash,caption=Execution of a robot simulation, label=impl-robot-run-simulation]
$ nohup roscore &
cd $PAL_ROBOT_DIR/local
$ output/bin/htmlDialog "http://localhost/..."
$ output/bin/guiServer --silent --noNavigation [--basestationFlash]
$ ../sources/bin/testStateMachine.sh --noNavigation
\end{lstlisting}

\ref{impl-robot-run-simulation} shows how these binaries are run.

The first version of the new Flango \cm could run in any of these three levels.
However, the final version runs in Google Chrome in the robot and there is no need to use this simulation.
In this case, interoperability is tested with \ac{ROS} topics in localhost and Google Chrome.

\subsection*{Tools}
The development of this project uses a number of tools to automate tasks and control de quality of the code to build a robust product.

\paragraph{\ac{IDE}} WebStorm\footnote{\url{http://www.jetbrains.com/webstorm}} is an \ac{IDE} based on IntelliJ IDEA designed to develop \acp{RIA}.
It provides integration with major frameworks, supports files with multiple languages (e.g. \ac{HTML} with embedded \ac{CSS} or JavaScript), provides tools for refactoring and on-the-fly code analysis (e.g. with JSHint/JSLint)

\paragraph{Karma Runner and Jasmine} Angular mixes very well with \ac{TDD}. 
It provides a framework (Jasmine) to write unit tests and a tool to automate them: Karma.
Tests can be executed in the background every time the code changes to make them fail early.
They run in a new instance of a browser and can be integrated in the \ac{IDE} to debug easily.
Additionally, it can be integrated with continuous integration systems, like Jenkins (used for the rest of the software in the robot).

\paragraph{Browsers and debugging} The first part of the project used a browser with \textbf{QtWebkit} both in karma and to launch the application.
There was one installation of Qt 4.8 and one of Qt 5.0.x, as Qt 5 was still experimental in the robot and had better support for \ac{CSS3} and \ac{HTML5}.
After this first stage, it was required to use a full browser: \textbf{Chrome} with Batarang and \textbf{Firefox} with Firebug.
\textbf{Batarang} is an extension for Chrome developed by the Angular Team to display the scope of elements, dependencies, etc.
\textbf{Firebug} is a widely-used extension for Firefox to debug web applications, modify \ac{HTML}, \ac{CSS} or \ac{JS} on the fly, set breakpoints, or even profiling.

\paragraph{SASS} \ac{SASS} is a scripting language that is interpreted into \ac{CSS}.
It has variables, mixins, inheritance, logical nesting blocks, arithmetic operations and functions among other features, and the code can be separated in several files that are compiled into a single \ac{CSS}.
It has two syntaxes: the original, and \ac{SCSS}, the newer one used in this project.
Listing \ref{sass-example} (L19) shows logical nesting and inheritance.

\lstinputlisting[label=sass-example,language=SASS, caption={SASS code example}]{src/sass-example.css}

\paragraph{Connection with robots} Robots are essentially two GNU/Linux computers with \ac{ssh} enabled.
\reem{H3} has one in the torso (\texttt{media}) where Flango is installed, and one in the lower half (\texttt{control}).
Listing \ref{impl-ssh} illustrates the connection with the media computer.

\begin{lstlisting}[columns=fixed,language=bash,caption=Connection with robots, label=impl-ssh]
$ ssh pal@reemh3-1m
Welcome to Ubuntu 12.04.1 LTS 
 (GNU/Linux 3.2.0-29-generic-pae i686)
...

pal@reemh3-1m:~$ ls /mnt_flash/srv/contents
admin       fonts      media
css         frontend   vpnUsersStatus.xml     
exports     img        vpnUsers.xml
flango.db   imports    
flango-gui  js         
flangoh     lib
\end{lstlisting}

\section{Directives and Modules}
Modules encapsulate directives.
Namespacing in JavaScript can be misleading (e.g. services should have unique names application-wide, even if they are in separate modules and a client only uses one of them).
To enforce isolation of components, this project puts all modules in separate closures \ref{impl-modules-closure}.

\begin{lstlisting}[language=JavaScript, caption=Closures to isolate modules, label=impl-modules-closure]
(function () {
    "use strict";
    /* Directives */
    var uicomponentsModule = angular.module('reemi.uicomponents');
    
    uicomponentsModule.directive('name1', function(...) {...});
    uicomponentsModule.directive('name2', function(...) {...});s
}());
\end{lstlisting}

Directives are the core of Flango \cm .
A simple directive that only does background logic (reading the \texttt{height} tag) is shown in listing \ref{impl-directive-height-example}.
It is restricted to elements (i.e. \lstinline$<fl:height>100</fl:height>$ triggers it but \lstinline$<fl:ui fl:height="100">$ does not)
\texttt{palProperties} is a service \textbf{injected} in directive.

\begin{lstlisting}[language=JavaScript,caption=flHeight Directive, label=impl-directive-height-example]
uiModule.directive('flHeight', function (palProperties) {
return {
        restrict: 'E',
        require: '^?flUi',
        link: function (scope, element, attrs, uiCtrl) {
           /* part of the behaviour */
        }
    };
});
\end{lstlisting}


\section{Services}
Angular Services can be declared with the \texttt{service()} constructor, with a \texttt{factory()} or, for complex scenarios, with a \texttt{provider()}.

\begin{lstlisting}[language=JavaScript,caption=Examples of service definition, label=impl-services-definition]
    palPropertiesModule.service('palProperties', function (palSettings) {});
    palSettingsModule.factory('palSettings', function ($http, $q...) {});
\end{lstlisting}

\lstinputlisting[label=impl-factory-reveal, language=javascript, caption={Example of Factory and Reveal Module Pattern}]{src/design-di-factory.js}


\texttt{service()} (listing \ref{impl-services-definition-service}) returns an object, whereas \texttt{factory()}  (listings \ref{impl-factory-reveal} and \ref{impl-services-definition-factory}) first creates it and then returns the instance.

\begin{lstlisting}[language=JavaScript,caption=Examples of service definition (service()), label=impl-services-definition-service]
palPropertiesModule.service('palProperties', function (...) {
    this.toCSS = function (scope, p) { ... };
    this.getProperty = function (scope, property, lang) { ... };
});
\end{lstlisting}

\begin{lstlisting}[language=JavaScript,caption=Examples of service definition (factory()), label=impl-services-definition-factory]
palSettingsModule.factory('palSettings', function (...) {
     var config, qs;
     var deferred = $q.defer();
     
     config = {generic: {}, app: {}, structure: {}, routes: {}};
     
     var getConfig = function () {...};
     var getGenericConfig = function () {...};
     ...
     
     return {
        'init': init,
        ...
     };
});
\end{lstlisting}

\section{Controllers}
\label{sec:controllers}
Controllers are key parts of any \ac{CRUD} application.
However, because Flango \cm does not know about the specific domain of the content application before running it, the only controllers that exist are those that receive events from the rendered \ac{GUI}, e.g. \texttt{ActionCtrl} (listing \ref{impl-action-ctrl}).
\textbf{Command Pattern} This controller implements the \textbf{command pattern} and uses JavaScript's methods \texttt{call} and \texttt{apply} to call a method by its name and passing in arguments \cite{Osmani:2012}.

\lstinputlisting[label=impl-action-ctrl,language=JavaScript, caption={ActionCtrl implementation}]{src/impl-controllers.js}

\section{Dependency Injection in JavaScript}
An instance of type \texttt{A} can create instances of type \texttt{B} (or use global variables). 
Both options couple classes \texttt{A} and \texttt{B} (e.g. hard-coded dependencies in \ref{impl-hardcoded-deps}).

\lstinputlisting[label=impl-hardcoded-deps, language=javascript, caption={Hard-coded dependencies}]{src/design-di-1.js}

\lstinputlisting[label=impl-nonhardcoded-deps,language=javascript, caption={Passing on the reference (without service locator)}]{src/design-di-handling-dependency-to-component.js}

However, in listing \ref{impl-nonhardcoded-deps} \texttt{MasterGreater} is not concerned with locating the \texttt{greeter} dependency, it is simply handed the \texttt{greeter} at runtime.
This is desirable, but it puts the responsibility of getting hold of the dependency on the code that constructs \texttt{MasterGreater}.

Angular applies the \emph{Dependency inversion} principle (\acs{SOLID}) intensively and has a service locator to find services and other dependencies.
Injectable components are created with a factory and are made available using a global injector.
They can be injected just by using their name.

A particular implementation of this principle is in the wrapper for roslibjs: the service \texttt{palROSBridge} in the application can be injected and its operations accept a callback function as a parameter enabling a decoupled design.

\lstinputlisting[language=javascript, caption={Injecting services in a controller}]{src/design-di-injection.js}

They can be obtained manually with the \texttt{\$injector}

\lstinputlisting[language=javascript, caption={Manual injection (e.g. in a test)}]{src/design-di-manual-injection.js}

Another example of inversion of control is implemented in \fref{sec:integration-with-ROS}.

\paragraph{A note about optimisation} To improve JavaScript performance and reduce network usage, the code can be minified before deployment.
This process reduces names of symbols, removes spaces, etc.
Dependency injection depends on the name of components. 
Angular provides a protection for minification: annotations.
\begin{lstlisting}[caption=Annotations to protect injection, language=javascript, label=impl-injection-annotation]
moduleA.run(['palSettings', '$location', '$rootScope', 'palROSBridge', 
    function (palSettings, $location, $rootScope, palROSBridge) {...} 
]);

\end{lstlisting}
Components can be injected as shown in listing \ref{impl-injection-annotation}.
The names are values of an array which, obviously, are not altered.

\section{Promises and Deferred Objects}
Remote calls in JavaScript are executed asynchronously.
The program in listing \ref{impl-async} generates the output shown in listing \ref{impl-async-out}.

\lstinputlisting[label=impl-async,language=javascript, caption={JavaScript Asynchronous calls}]{src/impl-async.js}
\lstinputlisting[label=impl-async-out,language=javascript, caption={JavaScript Asynchronous calls (output)}]{src/impl-async-out.js}

Asynchronous calls are normally handled with callbacks following the \ac{CPS}: control is passed explicitly in the form of continuation.

\lstinputlisting[label=impl-promises-call,language=javascript, caption={JavaScript: $printer$ will be called when $a$ returns}]{src/impl-promises-call.js}

\lstinputlisting[label=impl-promises-1,language=javascript, caption={JavaScript Callbacks}]{src/impl-promises-1.js}

Listing \ref{impl-promises-1} (called as in listing \ref{impl-promises-call}) shows a first approach\footnote{This is a simplification that might not cover all cases} to callbacks.
\texttt{\$http.get()} takes two parameters, the \texttt{success} and the \texttt{error} functions. 
They are executed after \texttt{\$http.get} has computed the return value, which is passed to the callbacks.
However, when there are nested or concurrent asynchronous calls, like the case of fetching the configuration from the \flangobe , callbacks become hard to maintain and readability of the code is dramatically affected.

\lstinputlisting[label=impl-promises-2,language=javascript, caption={JavaScript Callbacks (separate declaration)}]{src/impl-promises-2.js}

A better approach consists on extracting the body of the callback functions and using only pointers to these functions (listing \ref{impl-promises-2}).
This still does not resolve the problem of concurrency: the result of \texttt{a()} is not complete until \texttt{b()} and \texttt{c()} have completed (e.g. if the returned value of each function had to be aggregated into a \lstinline$var result = {a: [], b: []}$. 

\lstinputlisting[label=impl-promises-3,language=javascript, caption={Angular Promises}]{src/impl-promises-3.js}

Listing \ref{impl-promises-3} shows a real example from this project: a successful retrieval of the configuration parts for the application.
Angular's \texttt{\$q} service is an implementation of promises/deferred based on Kris Kowal's Q\footnote{\url{https://github.com/kriskowal/q}}.
A promise is \emph{resolved} only after all data has been fetched correctly from the \flangobe .
Promises can also be \emph{rejected} if necessary.
The code shows how control is passed: 
\begin{eqnarray}
init \rightarrow getConfig \rightarrow getGenericConfig \rightarrow getApplicationConfig \rightarrow getStructure \nonumber
\end{eqnarray}
The latest two operations could run concurrently but if \texttt{getApplicationConfig} failed, \texttt{getStructure} would have to be called again with \texttt{appId = "default"}

This promise is used to block the loading of the content applications until the palSettings service is correctly initialised.
Settings include security parameters like \texttt{perform\_callbacks} or \texttt{security\_level}, routes, \ac{URI} of the home screen, default language... that must not be undefined to guarantee a consistent boot strap of the application.

\paragraph{A note about security} Web or distributed systems normally have two levels of security: one client side, one server side.
Validation client-side simply helps filter or reduce the amount of remote calls (e.g. if the data of a form is not well-formatted, it should not be sent to the server because the client node already knows that it will fail).
Because JavaScript code client-side can be easily altered, the server side must not trust any input.
However, in this project safe input is assumed even server side because the code runs in a controlled embedded system and the \ac{UI} blocks access to the code.
Thus, if users modify the code client-side, changes do not affect any other user in the system either directly (e.g. a user gaining admin privileges) or indirectly (a user retrieving sensitive data).
Only engineers in the company have the ability to alter client-side code.
For example: an application runs with \lstinline$security_level = 1$ (user) because it should not be allowed to move the arms or make Reem say sentences aloud.
If client-side code is altered and \lstinline$security_level$ is set to \lstinline$2$ (admin), it will send the command to Qt (or publish it to a \ac{ROS} Topic, depending on the version of this project) and arms will move.
robotBehaviour assumes safe input from Flango \cm because the code can not be altered except by engineers in the company.

\section{Transformation to \acs{HTML}}
The \texttt{flUi} directive makes decoupling and reusing directives possible (listing \ref{impl-directive-ui}).
It removes the type attribute (\texttt{fl:type="base-button"} and creates an attribute with that value (\texttt{base-button}), which can trigger directives (\texttt{flBaseButton}, listing \ref{impl-directive-basebutton}).
Dependencies are \textbf{injected} by using their name in the parameters of the function.
The \texttt{controller} can be called from inner directives (e.g. flHeight in listing \ref{impl-ros-create-message}, the \texttt{compile} function manipulates the \ac{DOM} before it is replaced with anything and the \texttt{link} function reads in-line attributes (\texttt{flX}, \texttt{flY}...) in the last place, as the link function of inner elements is executed first.
Finally, it adds a class to a vector that is used to set \ac{CSS} style to the \ac{HTML} element.

\lstinputlisting[label=impl-directive-ui, language=JavaScript, caption={Summary of the flUi directive}]{src/impl-directive-ui.js}

After compiling, Angular finds a new directive (e.g.  \lstinline$<fl:ui fl:base-button ...>$ (\texttt{fl:ui} only runs functions if there is a \texttt{type} attribute, which was removed, and \texttt{flBaseButton} does have code to run).
flBaseButton (listing \ref{impl-directive-basebutton}) is a directive in the themes module.
It only links to a file that contains the template (listing \ref{impl-directive-basebutton-template}).
It is merely a \texttt{button} (a basic UI Component) with certain style set with the \ac{CSS} class \texttt{fl-base-button}.

\lstinputlisting[label=impl-directive-basebutton,language=JavaScript, caption={flBaseButton directive}]{src/impl-directive-basebutton.js}
\lstinputlisting[label=impl-directive-basebutton-template,language=xml, caption={flBasebutton directive template}]{src/impl-directive-basebutton-template.xml}

After this template is compiled, Angular finds another directive, \texttt{flButton} (listing \ref{impl-directive-button}), which has the logic to create an \ac{HTML} node for the actual button displayed on the screen.

\lstinputlisting[label=impl-directive-button,language=JavaScript, caption={flButton directive}]{src/impl-directive-button.js}

\lstinputlisting[label=impl-directive-basebutton-result,language=xml, caption={flBasebutton result}]{src/example-result-button.html}

At the end of the day, a button is displayed in the browser (listing \ref{impl-directive-basebutton-result}) using a link and \texttt{div}s with the \ac{CSS} class \texttt{fl-base-button}.
This style "includes" the style of a basic \texttt{button}.
Exclusive properties (like the position or size) are transformed to \ac{CSS} and added as in-line style in the element.

The result is a button like the one in the old version, one of the goals of the project.

\section{Integration with ROS}
\label{sec:integration-with-ROS}
Flango \cm is built as a component in \ac{ROS} that can interoperate with the rest of the system using \ac{ROS} topics.
Listing \ref{impl-ros-create-message} and \ref{impl-flango-ros-message} show the creation of a message for a ROS Topic in the software of the robot.

\begin{lstlisting}[caption=Creation of a ROS Message, label=impl-ros-create-message]
$ roscd pal_msgs
~/svn/stacks/pal_msgs$ roscreate-pkg pal_rb_flango_msgs
Created package directory pal_rb_flango_msgs
Created package file pal_rb_flango_msgs/Makefile
Created package file pal_rb_flango_msgs/manifest.xml
Created package file pal_rb_flango_msgs/CMakeLists.txt
Created package file pal_rb_flango_msgs/mainpage.dox

Please edit pal_rb_flango_msgs/manifest.xml and mainpage.dox 
to finish creating your package

~/svn/stacks/pal_msgs$ mkdir pal_rb_flango_msgs/msg

~/svn/stacks/pal_msgs$ vim pal_rb_flango_msgs/msg/rbFlango.msg
# Edit CMake

~/svn/stacks/pal_msgs/pal_rb_flango_msgs$ diff \
  CMakeLists.txt.orig CMakeLists.txt
20c20
< #rosbuild_genmsg()
---
> rosbuild_genmsg()

~/svn/stacks/pal_msgs/pal_rb_flango_msgs$ ls
CMakeLists.txt  CMakeLists.txt.orig  mainpage.dox  Makefile  
manifest.xml    msg

~/svn/stacks/pal_msgs/pal_rb_flango_msgs$ rosmake
[ rosmake ] rosmake starting...                                                                                                                                                                                     
[ rosmake ] No package specified.  Building ['pal_rb_flango_msgs']                                                                                                                                                  
[ rosmake ] Packages requested are: ['pal_rb_flango_msgs']                                                                                                                                                          
[ rosmake ] Logging to directory ~/.ros/rosmake/rosmake_output-2013...
[ rosmake ] Expanded args ['pal_rb_flango_msgs'] to:
['pal_rb_flango_msgs']                                                                                                                                         
[rosmake-0] Starting >>> pal_rb_flango_msgs [ make ]                                                                                                                                                                
[rosmake-0] Finished <<< pal_rb_flango_msgs [PASS] [ 6.49 seconds ]                                                                                                                                                 
[ rosmake ] Results:                                                                                                                                                                                                
[ rosmake ] Built 1 packages with 0 failures.                                                                                                                                                                       
[ rosmake ] Summary output to directory                                                                                                                                                                             
[ rosmake ] ~/.ros/rosmake/rosmake_output-20131125-202327                                                                                                                                         
~/svn/stacks/pal_msgs/pal_rb_flango_msgs$ 
\end{lstlisting}

\begin{lstlisting}[caption=Message type from robotBehaviour to Flango CM (rbFlango.msg), label=impl-flango-ros-message]
# message used by rb_flango
string           name
# Expected contents:
#   goTo
#   setLanguage
string           arg
\end{lstlisting}

After compiling the new message becomes available to the system.
ROS Bridge provides a \ac{JSON} interface to \ac{ROS}. Listing \ref{impl-flango-ros-rosbridge} shows it running on port 9090

\begin{lstlisting}[caption=ROS Bridge running, label=impl-flango-ros-rosbridge]
~$ nohup roscore &

~$ rosrun rosbridge_server rosbridge.py 
[INFO] [WallTime: 1385407762] Rosbridge server started on port 9090
[INFO] [WallTime: 1385408718] Client connected.  1 clients total.
[INFO] [WallTime: 1385408718] [Client 0] Subscribed to /web_gui_events
\end{lstlisting}

Flango \cm uses two topics: one for listening and one for publishing with different types of messages for convenience in the integration with robotBehaviour.
It is feasible with only one topic.
Topics are created the first time one publishes to them.
The example of listing \ref{impl-flango-ros-topic-pub} shows how a message is published.

\begin{lstlisting}[caption=Publishing to a topic, label=impl-flango-ros-topic-pub]
$ rostopic pub /web_gui_event pal_rb_flango_msgs/rbFlango 
"name: 'setLanguage' arg: 'de'" 
publishing and latching message. Press ctrl-C to terminate
\end{lstlisting}

\begin{lstlisting}[caption=Listening to a topic, label=impl-flango-ros-topic-echo]
~$ rostopic echo /web_gui_event
name: setLanguage
arg: de
---
\end{lstlisting}

When Flango \cm loads in the browser, a connection is established with the ROS Bridge server using web sockets.
\begin{lstlisting}[caption=Listening to a topic (Browser Console), label=impl-flango-ros-topic-browser]
subscribing /web_gui_event pal_rb_flango_msgs/rbFlango
controller got web_gui_event Object { name="setLanguage", arg="en"}
\end{lstlisting}

Flango \cm uses \texttt{roslibjs} to connect with ROS Bridge.
This library uses its own events system and needs to be wrapped in a service to work in the framework.

Essentially, roslibjs has 3 methods: \texttt{connect(host, port)}, \texttt{subscribe(name, messageType, callback)} (and \texttt{unsubscribe(name, callback)}) and \texttt{publish(name, message)}.

\begin{lstlisting}[caption=ROSLibJS connect method, label=impl-flango-ros-roslib-connect, language=javascript]
ros = new ROSLIB.Ros({
    url : 'ws://hostname:port'
});
\end{lstlisting}

Clients can subscribe to topics providing the name of the topic and the name of the type.
\texttt{ROSLIB.Topic} returns a listener that can wait for events.

\begin{lstlisting}[caption=ROSLibJS subscribe method, label=impl-flango-ros-roslib-subscribe, language=javascript]
var listener = new ROSLIB.Topic({
    ros : ros,
    name : '/topic_name',
    messageType : 'std_msgs/String'
});
listener.subscribe(function(message) { /* callback */ } );
\end{lstlisting}

\textbf{Inversion of control} \texttt{palROSBridge} service interfaces to \texttt{ROSLib}.
Services can be injected and add logic specific for Flango \cm .
To make the subscribe callback available to the client instance, one of the parameters of \texttt{palROSLib::subscribe()} is a function:

\begin{lstlisting}[language=JavaScript, caption=palROSLib subscribe method, label=impl-flango-ros-palroslib-subscribe]
var listeners = {};
var subscribe = function (name, messageType, callback) {
    listeners[name] = new ROSLIB.Topic({...});
    listeners[name].subscribe(function(message) {
        callback(message);
    });
};
\end{lstlisting}

\textbf{Reveal Pattern} This pattern allows for a clearer way of writing JavaScript modules.
Private methods are in a private closure and only a few are exposed in the returned object as a public \ac{API} with pointers to the internal operation.

\begin{lstlisting}[language=JavaScript, caption=Reveal Pattern in palROSBridge, label=impl-flango-ros-palroslib-reveal]
palROSBridgeModule.factory('palROSBridge', function () {
    /* init logic*/
    var ros, listeners = {};
    var connect = function(host, port) {...};
    ...
    // ROSLIB.Topic.subscribe creates a list of subscribers.
    // this is just delegating the task
    var subscribe = function (name, messageType, callback) {...};
    var unsubscribe = function (name, callback) {...};
    // Reveal methods as a public API
    var service = {
        'connect': connect,
        'subscribe': subscribe,
        'unsubscribe': unsubscribe,
        ...
   };
   return service;
});
\end{lstlisting}