\chapter{Implementation}
\cite{Stefanov} \cite{Crockford} \cite{Darwin} \cite{AngularJSGuide} 
This chapter describes the environment to develop and run Flango \cm and the implementation of the key features.

\section{Development environment set-up}
webstorm, chrome+batarang, firefox+firebug, karma, qtwebkit, qt5 and qt4, linux, ssh to access the robots and basestation, python, virtual environments, svn, robot code, run levels (htmldialog, guiServer, statemachine (WARNING: CONFIDENTIAL?): motivation and benefits.

The Flango \cm belongs to a bigger project.
The development environment is adapted to this situation to ensure a correct integration.
Moreover, the project uses tools to run unit tests, generate code, etc.

\subsection{Integration}
The files of Flango \cm live in the server as part of the \flangobe .
The backend is developed with Django and requires a set of Python libraries that might not be available in the system.
To overcome this situation, Flango is developed in a virtual environment and files are made available to clients with the built-in web server in Django, although Basestation uses Apache2 and the robot Lighttpd.
All the code is in a subversion repository.

\begin{lstlisting}[caption=lalala, label=virtual-env]
$ mkvirtualenv local-flango
$ workon local-flango
$ pip install --upgrade -r requirements.txt
\end{lstlisting}

Virtual environments have pre and postactivate scripts to alter the default behaviour.
Flango needs to modify environment variables to make \texttt{local-flango} live in the local copy of the repository, use a database, set aliases, etc.

After the virtual environment is ready, it is easier to change between branches.
Django is one of the packages available in the virtual environment and has a built-in web server to develop that runs on port 8000.

\begin{lstlisting}[caption=Django web server, label=virtual-env-server]
~/contentMgmtBranches/trunk/backend$ django runserver
Validating models...

modeltranslation: Registered 0 models for translation () [pid: 5286].
0 errors found
Django version 1.4.3, using settings 'flango.settings'
Development server is running at http://127.0.0.1:8000/
Quit the server with CONTROL-C.

[25/Nov/2013 15:17:58] "GET /static/flangoh/flangoh/app/index.html HTTP/1.1" 200 2317
\end{lstlisting}

The file structure of the whole project is shown in listing \ref{impl-flango-files-layout}.
Folders \texttt{build}, \texttt{packaging} and \texttt{conf} contain scripts to build debian packages to deploy the software on the robot and basestation.
\texttt{src} contains the \flangobe , which includes the \flangofe (and the \se).
The project of this thesis, Flango \cm is in the \texttt{static} folder, that contains files served without any processing server-side.

\begin{lstlisting}[caption=Flango files layout, label=impl-flango-files-layout]
~/contentMgmtBranches/trunk/backend$ ls
build  conf  packaging  requirements.txt  scripts  src  static
\end{lstlisting}

The browser can open any file in the \texttt{static} folder.
Alternating between the new and the old version only requires to load \texttt{frontend/index.html} (\flash) or \texttt{flangoh/flangoh/app/index.html} (AngularJS).

\begin{lstlisting}[caption=Flango files layout, label=impl-flango-files-layout]
~/contentMgmtBranches/trunk/backend/static$ ls
admin       fonts     lib
css         frontend  media
exports     img       vpnUsersStatus.xml
flango-gui  imports   vpnUsers.xml
flangoh     js        
\end{lstlisting}

The development computer also has a working copy of the code for the robot and Basestation to run simulations and test the new \cm  without using robot time, a scarce resource.

There are 3 run levels:
\begin{enumerate}
\item \textbf{htmlDialog}. A basic Qt dialog with a QtWebKit widget
\item \textbf{guiServer}. A version with more features enabled, like the sound server , navigation, etc.
\item \textbf{stateMachine}. The complete suite with a testing state machine, Qt, sound server, etc.
\end{enumerate}

An \ac{XML} file defines the \ac{URL} to load.

The first version of the new Flango \cm could run in any of these three levels.
However, the final version runs in Google Chrome in the robot and there is no need to use this simulation.
In this case, interoperability is tested with \ac{ROS} topics in localhost.

\subsection{Tools}

\section{Examples (application)}
this is gonna be too big. should examples be included somewhere else?

include examples about: a service, a simple directive, a controller (and the execution flow?).

include directive examples: normal (back-button), with isolated components (subscreen), and entities (data and loop)

\section{Integration with ROS}
Flango \cm is built as a component in \ac{ROS} that can interoperate with the rest of the system using \ac{ROS} topics.

Listing \ref{impl-ros-create-message} and \ref{impl-flango-ros-message} show the creation of a message for a ROS Topic in the software of the robot.


\begin{lstlisting}[caption=Flango files layout, label=impl-ros-create-message]
$ roscd pal_msgs
~/svn/stacks/pal_msgs$ roscreate-pkg pal_rb_flango_msgs
Created package directory /home/eduardgamonal/svn/stacks/pal_msgs/pal_rb_flango_msgs
Created package file /home/eduardgamonal/svn/stacks/pal_msgs/pal_rb_flango_msgs/Makefile
Created package file /home/eduardgamonal/svn/stacks/pal_msgs/pal_rb_flango_msgs/manifest.xml
Created package file /home/eduardgamonal/svn/stacks/pal_msgs/pal_rb_flango_msgs/CMakeLists.txt
Created package file /home/eduardgamonal/svn/stacks/pal_msgs/pal_rb_flango_msgs/mainpage.dox

Please edit pal_rb_flango_msgs/manifest.xml and mainpage.dox to finish creating your package
~/svn/stacks/pal_msgs$ mkdir pal_rb_flango_msgs/msg
~/svn/stacks/pal_msgs$ vim pal_rb_flango_msgs/msg/rbFlango.msg
# Edit CMake
~/svn/stacks/pal_msgs/pal_rb_flango_msgs$ diff  CMakeLists.txt.orig CMakeLists.txt
20c20
< #rosbuild_genmsg()
---
> rosbuild_genmsg()
~/svn/stacks/pal_msgs/pal_rb_flango_msgs$ ls
CMakeLists.txt  CMakeLists.txt.orig  mainpage.dox  Makefile  manifest.xml  msg
~/svn/stacks/pal_msgs/pal_rb_flango_msgs$ rosmake
[ rosmake ] rosmake starting...                                                                                                                                                                                     
[ rosmake ] No package specified.  Building ['pal_rb_flango_msgs']                                                                                                                                                  
[ rosmake ] Packages requested are: ['pal_rb_flango_msgs']                                                                                                                                                          
[ rosmake ] Logging to directory /home/eduardgamonal/.ros/rosmake/rosmake_output-20131125-202327                                                                                                                    
[ rosmake ] Expanded args ['pal_rb_flango_msgs'] to:
['pal_rb_flango_msgs']                                                                                                                                         
[rosmake-0] Starting >>> pal_rb_flango_msgs [ make ]                                                                                                                                                                
[rosmake-0] Finished <<< pal_rb_flango_msgs [PASS] [ 6.49 seconds ]                                                                                                                                                 
[ rosmake ] Results:                                                                                                                                                                                                
[ rosmake ] Built 1 packages with 0 failures.                                                                                                                                                                       
[ rosmake ] Summary output to directory                                                                                                                                                                             
[ rosmake ] /home/eduardgamonal/.ros/rosmake/rosmake_output-20131125-202327                                                                                                                                         
~/svn/stacks/pal_msgs/pal_rb_flango_msgs$ 
\end{lstlisting}

\begin{lstlisting}[caption=Message type from robotBehaviour to Flango CM (rbFlango.msg), label=impl-flango-ros-message]
# message used by rb_flango
string           name
# Expected contents:
#   goTo
#   setLanguage
string           arg

\end{lstlisting}

After compiling the new message becomes available to the system.
ROS Bridge provides a \ac{JSON} interface to \ac{ROS}. Listing \ref{aaa} shows it running on port 9090

\begin{lstlisting}[caption=ROS Bridge running, label=impl-flango-ros-rosbridge]
~$ nohup roscore &
~$ rosrun rosbridge_server rosbridge.py 
[INFO] [WallTime: 1385407762.165122] Rosbridge server started on port 9090
[INFO] [WallTime: 1385408718.553601] Client connected.  1 clients total.
[INFO] [WallTime: 1385408718.559174] [Client 0] Subscribed to /web_gui_events

\end{lstlisting}

Flango \cm uses two topics: one for listening and one for publishing with different types of messages for convenience in the integration with robotBehaviour.
It is feasible with only one topic.
Topics are created the first time one publishes to them.
The example of listing \ref{aaa} shows how a message is published.
\begin{lstlisting}[caption=Publishing to a topic, label=impl-flango-ros-topic-pub]
$ rostopic pub /web_gui_event pal_rb_flango_msgs/rbFlango "name: 'setLanguage'
arg: 'de'" 
publishing and latching message. Press ctrl-C to terminate
\end{lstlisting}

\begin{lstlisting}[caption=Listening to a topic, label=impl-flango-ros-topic-echo]
Listening to a topic
~$ rostopic echo /web_gui_event
name: setLanguage
arg: de
---
\end{lstlisting}

When Flango \cm loads in the browser, a connection is established with the ROS Bridge server using websockets.
\begin{lstlisting}[caption=Listening to a topic (Browser), label=impl-flango-ros-topic-browser]
Browser console: subscribing /web_gui_event pal_rb_flango_msgs/rbFlango
Browser console: controller got web_gui_event Object { name="setLanguage", arg="en"}
\end{lstlisting}

FIXME add here parts of the code of the service
