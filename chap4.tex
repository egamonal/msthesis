\chapter{Requirements}
Requirements can be gathered using interviews with customers, users and stakeholders, with observation, questionnaires, prototyping or even from an existing application.
In this project requirements are obtained mainly from the current implementation of the software (\fref{sec:reengineering}) and from interviews with the developer of the old version.

\section{Stakeholders}
maybe this subsection is not needed

\section{Functional requirements}

\subsection{Ready-made solutions}
Because this is a very specific software, there are no off-the-shelf solutions available that meet the requirements of the project.
However, there are libraries and frameworks that can be reused:

TODO: Add description or a few more details (?)

\begin{enumerate}
    \item JavaScript frameworks
    \begin{itemize}
        \item MVC: AngularJS, Spine, Ember.js, Knockout.js, Sprout, Google Closure
        \item Oriented to Web Components: AngularJS, Polymer
        \item BackBone.js
    \item Testing: Jasmine, qUnit, phantomJS   
    \end{itemize}
    \item JavaScript libraries
    \begin{itemize}
        \item jQuery and plug-ins: Bigvideo.js, jQuery-ui, jquery-mobile , jquery-kinetic  
        \item Hammer.js, jGestures , iScroll, swipe.js
    \end{itemize} 
    \item CSS Tools: LessCSS, SASS, Bootstrap
\end{enumerate}

\subsection{Glossary}

\subsection{Domain properties and hypothesis}
\subsection{Constraints}
Certain constraints are defined before the beginning of the project:
\begin{enumerate}
    \item Technology: the project has to be implemented using modern web technologies (HTML5, JavaScript, CSS).
    \item Legal: It has to be ready to be open source
\end{enumerate}


\subsection{Functional requirements}
\begin{enumerate}
    \item Render UI Components with HTML. UI components, currently defined with XML, have to be eventually transformed to HTML so that a browser can render them.
    \begin{enumerate}
        \item Support for multiple languages via the parameter lang.
        \item Components 
        \begin{enumerate}
            \item Basic components screen, subscreen, textarea, img, button, group, video, showreel, swf, layout, qr, group\_button
            \item Theme components
            \item Properties inline or in tags. x, y, width, height, scrolly, scrollx, loop, loopy, loopx, src, style, caption
            \item Action properties onclick, onload, action, param
        \end{enumerate}
        
    \end{enumerate}
    \item Themes. Support Theming. Different look feel
    \item Internal navigation (subscreens)
    \item Settings management
    \item External communication with the robot and backend
    \item Entities support
    \item Screen saver / Sessions
\end{enumerate}


do we need a section somewhere about DSL? is our XML a DSL?
basically, make it work and make it work fast. test. deploy it. make it compatible with backend

\section{Non Functional Requirements}
\begin{enumerate}
    \item Target browser: QtWebKit (Qt 5.0.x) (FIXME! maybe chrome)
    \item Extensibility: The software has to be extensible. The design has to allow adding new features in the future.
    \item Robustness: the software has to be reliable and has to be delivered with a comprehensive test suite. It should be compatible with Jenkins and continuous integration.
    \item Hardware: the software has to perform smoothly in Reem H2 and a multitouch screen.
    \item UX: the time to change to another screen in the rendered contents application should be less than 0.5s. Media contents (images, videos) have to be ready in less than 1s after a screen change. 
\end{enumerate}
