\chapter{Conclusion}
\section{Contributions}
This thesis is emintently a practical work with a clear goal: reengineer Flango \cm with web technology.
The new version of the software works and is ready for production.
It covers 80\% of the most common features and can substitute safely the old \flash system.

%The project is unique and new in the company and solves a need.

This thesis was conceived to apply the principles and techniques acquired not only in this master's course but in the BSc degree as well related to 


Initially, the project was planned in terms of scope, schedule and budget.
Secondly, requirements were gathered from exploring the old version and discussing about the system with the previous developer.

After this, the specification of the system was made combining \ac{TDD} and design by contract for the behaviour model, and a class diagram for the conceptual model.
This part was kept as technology agnostic as possible and emphasised the first three steps in a reengineering process: inventory analysis, documentation restructuring and reverse engineering.

Then, the system was designed internally taking into account software engineering principles (\ac{GRASP} and \ac{SOLID}).
It described the architecture of the software (\ac{MVC}), the patterns (dependency injection, command, factory, revealing module AND FIXME REMOTE FACADE), the strategy to interoperate with the robot's system and with the back-end.
Several diagrams for a static and a dynamic views were made: class and packages diagrams, sequence diagrams of relevant operations and deployment diagrams.

Finally, the software was implemented with Angular (JavaScript), HTML, SASS and ROSLibJS, and minor changes were made in the back-end (Python).

The testing strategy had \ac{TDD} at the core all throughout the development of the project.
Unit tests were used as executable specifications and helped design better components that are less coupled and more cohesive.

Despite the fact that this thesis sorts chapters as if it had been developed using the waterfall method, the software was designed in small iterations and milestones.

FIXME and now a paragraph about what I learnt in the master's to impress everybody

\section{Challenges}
 FIXME MAYBE NOT NECESSARY
challenges: tech unkown, soa, fitting xml for flash in web

The project had challenges of diverse nature that were overcome succesfully with patience, attention to details, creativity and logic.

One of the biggest challenges was that the old technology and product were unknown and not documented at the beginning.
Understanding the role of the application in the system was crucial to develop a successful new product.
Only in the last weeks of development it was noticed that the application could have been built as a service to make interoperation with ROS easier.
However, the internal design was flexible enough to adapt to changes and take advantage of the framework.

Another big challenge was to adapt to \ac{XML} and a back-end designed for an ActionScript application.
Choosing the technology correctly was critical to minimise the work.




\section{Future Work}

davsdfvbsdf