% $Log: abstract.tex,v $
% Revision 1.1  93/05/14  14:56:25  starflt
% Initial revision
% 
% Revision 1.1  90/05/04  10:41:01  lwvanels
% Initial revision
% 
%
%% The text of your abstract and nothing else (other than comments) goes here.
%% It will be single-spaced and the rest of the text that is supposed to go on
%% the abstract page will be generated by the abstractpage environment.  This
%% file should be \input (not \include 'd) from cover.tex.

This project aims at reengineering a content manager for humanoid robots with web technology at \company in order to abandon the current \flash implementation.
This software runs in the robot, displays content applications and handles user interaction.

In the first part, this thesis describes the boundaries and context of the project and provides a general overview of related topics that support this work.

In the second part, this thesis formally addresses the reengineering of the software based on Pressman's work, in 6 following steps: software inventory, documentation restructure, reverse engineering, code and data restructuring and forward engineering.

First it identifies the constraints of the project as well as its functional and non-functional requirements.

Secondly, it presents the specification emphasising the first 3 steps.
This document includes a conceptual model, system use cases and sequence diagrams.

Thirdly, it describes the internal design of the system in the context of the last 3 steps.
It starts by highlighting the system's architecture, its context and the software patterns.

Fourthly, it provides a static view with class and packages diagram, a dynamic view with sequence diagrams and a physical view with a deployment diagram.

The following chapter provides an overview of the most relevant parts of the implementation with code examples.

Finally, it outlines the testing strategy and explains the testing system's implementation.

In conclusion, this master's thesis addresses the reengineering of the content manager and develops the new system with accuracy, creativity and consistency, applies a systematic approach and uses proven techniques like software patterns and widely-known architectures.

