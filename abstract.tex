% $Log: abstract.tex,v $
% Revision 1.1  93/05/14  14:56:25  starflt
% Initial revision
% 
% Revision 1.1  90/05/04  10:41:01  lwvanels
% Initial revision
% 
%
%% The text of your abstract and nothing else (other than comments) goes here.
%% It will be single-spaced and the rest of the text that is supposed to go on
%% the abstract page will be generated by the abstractpage environment.  This
%% file should be \input (not \include 'd) from cover.tex.

Software products do not degrade with time or with external factors, but they are continually corrected and extended.
Using third-party packages is a common practice and creates a dependency with a vendor, who might stop supporting a product.
After a certain amount of time, changes become harder to implement and reengineering might be necessary.


This project aims to reengineer a content manager for humanoid robots with web technology at \company in order to abandon the current \flash implementation.
This software runs in the robot, displays content applications and handles user interaction.
Users create contents with an external \acs{RIA}. 
These might contain buttons, images or text, and have an \acs{XML} representation.
The content manager in the robot loads the content, interprets that \acs{XML} in order to build a \acs{GUI} with \acs{HTML}, and interoperates with the robot's system from the browser.

This thesis formally addresses the reengineering of the software based on Pressman's work, in 6 successive steps: software inventory, documentation restructuring, reverse engineering, code restructuring, data restructuring, and forward engineering.

Firstly, it identifies the functional and non-functional requirements of the project.
Secondly, it presents the specification emphasising the first three steps, including a conceptual model, system use cases and sequence diagrams.
Thirdly, it describes the internal design of the system in the context of the last three steps.
It starts by highlighting the system's architecture, its context and the software patterns.
Then, it provides a static view with class and packages diagrams, a dynamic view with sequence diagrams and a physical view with a deployment diagram.
Fourthly, it illustrates relevant parts of the implementation with code examples.
Lastly, it outlines the testing strategy and implementation.

In conclusion, this master's thesis addresses the reengineering of the content manager and develops the new system with accuracy, creativity and consistency. 
It applies a systematic approach and uses proven techniques like software patterns and a widely-known architecture.

