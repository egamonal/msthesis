\chapter{Project Management}
This chapter describes the planning of this project in terms of scope, schedule, cost and risk.
The execution and deviations are shown at the end of this chapter.

\section{Scope}
\label{sec:scope}
The system has two parts: the basestation (with the backend) and the screens renderer in the robots (the frontend).
This project reengineers the frontend with web technologies but not the backend, that remains in Django and Adobe Flash.

The new frontend accepts a contents application (GLOSSARY??) as input, generated with the Screens Editor in the backend, and outputs the screen rendered with web elements.
Contents applications have configuration files that define the theme, the location of binary resources (images, videos, etc), the default language and other settings. 
This project can read the configuration files and set up the front-end.
A screen of a contents application is an \ac{XML} file. 
This project accepts valid syntax generated with the Screens Editor for a comprehensive set of components (back-button, base-button, image, video...) and their properties (width, x, y, background...), and generates HTML5 that renders them as defined.

Developing an intense testing strategy and is part of this project.
The final artifact is this master thesis that describes the system and the rationale that guides all decisions.

\section{Schedule}
This project has 4 phases that match the typical 4 groups of processes: initiating, planning, executing (and monitoring + control) and closing.
The execution has 4 deliverables: viability and proof of concept, iteration 1 (basic program), iteration 2 (extended program), iteration 3 (improvements).

\subsection{Phase 1: Initiating}
\paragraph{April, 1 -- April 4}
The initiating processes determine the goals of the project and the scope.
This stage was partially done before the beginning of this thesis to ensure its viability in the company.
It was agreed that the project would reengineer the current system using web technologies (see \fref{sec:scope})

\subsection{Phase 2: Planning}
\paragraph{April, 5 -- April 6}
The first draft of the plan is a rough estimation of the tasks length and a definition of the big milestones.
Because there is only one engineer there are no concurrent tasks and free floats are always zero.
Other activities of this phase include an estimation of the resource requirements for upcoming tasks, developing the budget (\fref{sec:budget}) and risk planning (\fref{sec:risk}).
See the Gantt chart in (CROSS REFERENCE HERE) 

\subsection{Phase 3: Execution}
\paragraph{April, 7 -- August 28}
The execution phase focuses on building the product. 
A web engineering process should be incremental, with frequent changes and short iterations \cite{Kappel:2006}.

\begin{table}[ht]
    \centering
    \caption{Milestones and deliverables of the execution phase}
    \label{tab:milestones}
    \begin{tabularx}{\linewidth}{| X | X |}
    \hline
    Milestone & Deliverables \\
    \hline
    Proof of concept & Research on frameworks and tools to develop the project. Choose one. Build a minimal working example to illustrate key aspects of the project. \\ \hline
    Iteration 1 & Build a prototype comprehensive in number of features but simple in their implementation to validate the technology. \\ \hline
    Iteration 2 & Extend the prototype of Iteration 1 to complex cases of the features. \\ \hline
    Iteration 3 & Add special new features. \\
    \hline
    \end{tabularx}
\end{table}

This phase has 4 big milestones, one per iteration (\fref{tab:milestones}). 
The outcome of each iteration is a product with a set of stable features and the corresponding unit tests which, in turn, serve for the purpose of documentation.

\subsection{Phase 4: Closing}
\paragraph{August, 29 -- September 14}
Prepare package with a stable set of features, presentation and documentation.

\section{Cost}
\label{sec:budget}
The estimated length of the project is $105 days \times 8 \euro{} \times 8 hours = 6720 \euro{} $

FIXME check cost control (amortized? price?)

\begin{table}[ht]
    \centering
    \caption{Budget}
    \label{tab:budget}
    \begin{tabularx}{\linewidth}{| X | X |}
    \hline
    Labour & 6720\euro{} \\ \hline
    \multirow{3}{*}{Hardware} 
        & Desktop Computer (amortized) 250\euro{} \\ % check this
        & 22 Monitor (amortized) 50\euro{} \\
        & Reem H3 testing time 150\euro{} \\ \hline   
    \multirow{2}{*}{Software}
        & JetBrains WebStorm license 89.10\euro{} \\ % check this    
        & GNU/Linux 0\euro{} \\ \hline
     Overall & 7259.10\euro{} \\ 
     \hline
    \end{tabularx}
\end{table}

There are some infrastructure related costs excluded from the budget because they are part of the office maintenance monthly expenses. This is the case of subversion servers, network, power, backups, etc. 

\section{Risk Analysis}
\label{sec:risk}
for each risk: title, description, mitigation plan, action plan

\section{Execution}
progress, deviations, final cost
