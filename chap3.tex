\chapter{Plan}
check section name (roadmap?). check contents.
description of the dev process: research, write small pieces of code, intense unit testing, refactoring, TDD...

This chapter describes the planning of this project in terms of scope, time, budget, methodology and risk.
The execution and deviations are shown at the end of this chapter.

\section{Methodology: Test-Driven Development}
The software written for this thesis is executed in a complex environment.
Testing it properly and ensuring the quality is crucial to avoid unexpected behaviour in one of the most visible parts of this system, the screen.
Crashes and \ac{UI} flaws can lead to triggering wrong external calls to other components of the Qt system (e.g. displaying a Qt dialog different than expected, not making the call, sending it with bad parameters...), blocking access to features or seriously affecting the \ac{UX}.
Reengineering this system has, at least, two sources of potential errors: 
undocumented features in the original application (unknown, documented wrongly, partially documented or not described at all) and integration of the new software in the current system.

\ac{TDD} is based on having numerous and very short iterations with these steps:
\begin{enumerate}
    \item Add an initial (failing) test
    \item Run all tests and see if the new one fails
    \item Write the minimum amount of code to pass the test
    \item Run the automated tests and see them succeed
    \item Refactor the code. Conform to standards and best practices
\end{enumerate}
This methodology is specially useful in the implementation of this project for two reasons: 
the Google AngularJS framework is extremely friendly with unit testing and E2E tests thanks to Karma Runner, and the company has the infrastructure to incorporate the project in a continuous integration system with Jenkins.


\section{Phases}
4 phases: proof of concept, first iteration, second, report (?). gantt


\section{Budget}
numbers here.
\section{Risk Analysis}

\section{Execution}
progress, deviations, final cost


\chapter{Requirements}
\section{Stakeholders}
maybe this subsection is not needed
\section{Scope}
scope of the project. what will be done and what will not. e.g backend.basestation, old xml syntax.

\section{Functional requirements}
do we need a section somewhere about DSL? is our XML a DSL?
basically, make it work and make it work fast. test. deploy it. make it compatible with backend

\section{Non Functional Requirements}
intense testing, robust application, modern standards, easy to maintain, specific platform, specific hardware, touchscreen, UI/UX/usability

\chapter{Specification}
uhms, I don't think this section will have interesting diagrams.
should it contain the theory? DSL, XML (vocabulary, grammar, motivations), intercommunication with other systems (config files, structure.xml...)

relate to reengineering (R Pressmann): abstraction, procedures, etc.

Users: no users (well, a "virtual" user does exist). it's a meta-app that reads apps in xml and renders them in html. define input and output.

contracts: no contracts. TDD as opposed to design by contract??? check literature

Use Cases: one per directive, settings and entities?? users: the system.

\chapter{Design}
internal design of the application. this should be the fun part.
\section{Technology}
angularJS (teach the browser new syntax), javascript, html5, Qt, QtBrowser and its limitations (JS engine vs chrome v8 or firefox's).

\section{Architecture}
MVC. client-side. JSON to talk to an API. xml files as angularjs partials, hacks to make it work (routes defined manually, one controller for all of them, dirty entities). scopes and their specific use here (compare to regular use)

dependency injection and angularJS (and literature: fowler)

\section{Static View}
the model, the view and the controllers. angualar modules, services, filters, controllers...
talk about the data model and \$rootScope ?

\section{Dynamic view}
the flow: user creates an application, [robot sync], my app reads it and generates HTML output.

specific flow of an app. bootstrap, the compile-link phase, push classes, push styles, get url params, etc.

\section{Physical view}
robots, backend.basestation, wi-fi, vpn...

\section{External view}
there isn't! it's a backend app to create a frontend app.

\section{should there be any more sections?}

\chapter{Implementation}
\section{Development environment set-up}
webstorm, chrome+batarang, firefox+firebug, karma, qtwebkit, qt5 and qt4, linux, ssh to access the robots and basestation, python, virtual environments, svn, robot code, run levels (htmldialog, guiServer, statemachine (WARNING: CONFIDENTIAL?): motivation and benefits.

\section{Examples (application)}
this is gonna be too big. should examples be included somewhere else?

include examples about: a service, a simple directive, a controller (and the execution flow?).

include directive examples: normal (back-button), with isolated components (subscreen), and entities (data and loop)

\chapter{Testing}
i'm gonna enjoy myself here.
TDD, karma, unit tests, specification with tests, safe refactoring, integration of karma in webstorm, tweak karma to run in Qt...
quality factors of a webapp: kappel


\chapter{Future Work and Conclusions}
\section{Future Work}
\section{Conclusions}