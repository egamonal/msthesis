\chapter{Design}
explain the internal design of the new application.
internal design of the application. this should be the fun part.
\section{Technology}
angularJS (teach the browser new syntax), javascript, html5, Qt, QtBrowser and its limitations (JS engine vs chrome v8 or firefox's).


\section{Architecture}
context: comm with rob behaviour and rostopics (SOA).
MVC. client-side. JSON to talk to an API. xml files as angularjs partials, hacks to make it work (routes defined manually, one controller for all of them, dirty entities). scopes and their specific use here (compare to regular use). what about MVVM?

common patterns:
dependency injection and angularJS (and literature: fowler)

\subsection{Context}
Flango Content Manager is a piece in the robot.
\ac{ROS} is a framework for robot software development that provides operating system-like functionality on a heterogeneous computer cluster. 
It was originally developed in 2007 as part of the Stanford AI Robot (STAIR) project with the goal of providing an architectural framework supporting modular, tool-based development for robotic software.
It has a loosely coupled architecture with nodes, messages, topics and services, similar to \ac{SOA}.

DIAGRAM HERE:_ rosmaster (service registry), services, topics

Nodes are processes that perform computation, a "software module". 
Nodes communicate with Messages, strictly typed data structures . There are some available by default, like $std_msgs/String$ and other primitive types.
Nodes subscribe and publish to topics, identified by a name (string).
Services are pairs of request/response messages offered in a node designed for simple synchronous transactions.
FIXME add ActionServers???


http://www.willowgarage.com/sites/default/files/icraoss09-ROS.pdf

The robot has two parts: robotBehaviour and Stacks.
Part of the code of the robot (robotBehaviour) does not follow the \ac{ROS} workflow but still uses Topics to communicate with the Stacks, the part of the project that does use \ac{ROS}.
In Stacks there are Servers and ActionServers that provide Services.
Flango is a component in this system that interacts with robotBehaviour using \ac{ROS} Topics.
It is not built as a service

OR MAYBE DIAGRAM HERE:_ rosmaster (service registry), services, topics, FLANGO


As of 2008,
development continues primarily at Willow Garage, a robotics
research institute/incubator, with more than twenty institutions
collaborating in a federated development model. 

ROS i SOA.
robotbehaviour
flango as a (bad) service.


\section{Static View}
the model, the view and the controllers. angualar modules, services, filters, controllers...
talk about the data model and \$rootScope ?

\section{Dynamic view}
the flow: user creates an application, [robot sync], my app reads it and generates HTML output.

specific flow of an app. bootstrap, the compile-link phase, push classes, push styles, get url params, etc.
examples of using the properties, generating the settings, fetching things from the server, rostopics...

\section{Physical view}
robots, backend.basestation, wi-fi, vpn...

\section{External view}
there isn't! it's a backend app to create a frontend app.

\section{should there be any more sections?}
something more